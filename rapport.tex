\documentclass{article}
\usepackage[utf8]{inputenc}
\usepackage[francais]{babel}
\title{Compilateur \texttt{MiniC++}}
\author{Corentin Cadiou, Ken Chanseau Saint-Germain}
\date{29 novembre 2013}
\begin{document}
\maketitle
Lors de ce projet, nous avons eu à réaliser un compilateur pour un
sous-ensemble de C++, appelé MiniC++.

\section*{Compte-rendu des fonctions du compilateur}
Les analyses lexicale, sémantique et de type ont un comportement
correct vis-à-vis des conventions de MiniC++.

Pour la production de code, notre compilateur a le comportement attendu pour :
\begin{itemize}
\item l'affichage;
\item l'arithmétique;
\item les variables locales (le test \texttt{while.cpp} échoue, cela vient
  d'une mauvais allocation de la mémoire);
\item les pointeurs;
\end{itemize}

Notre compilateur a un comportement acceptable mais parfois buggué
avec :
\begin{itemize}
\item les fonctions :
  \begin{itemize}
    \item empilement et recherche correcte des arguments;
    \item valeur de retour;
    \item dépilement correct, mais parfois \texttt{\$sp} ne revient pas à l'endroit attendu;
  \end{itemize}
\item les références :
  \begin{itemize}
  \item quelques problèmes sûrement en rapport avec les fonctions;
  \item gestion correcte du référencement et du déréferencement;
  \end{itemize}
\item les objets :
  \begin{itemize}
    \item gestion correcte de l'allocation en mémoire, y compris pour
      les classes héritées;
    \item gestion correcte de la recherche de méthode (statique) et de
      constructeur;
    \item quelques problèmes d'accès aux champs, car les variables
      locales sont mal stockées : il semblerait qu'elles aient été
      stockées dans le mauvais sens (pas en descandant sur la pile
      mais en montant);
    \item pas de gestion des méthodes virtuelles, mais héritage des
      classes des parents;
  \end{itemize}
\end{itemize}

\emph{NB : on compilera avec amusement un programme qui nécessite
  d'utiliser une cinquantaine de label.}
\end{document}
